\hypertarget{index_intro}{}\section{Introduction}\label{index_intro}
Projet Damibin réalisé par \+: ~\newline

\begin{DoxyItemize}
\item B\+E\+N\+S\+O\+U\+S\+S\+A\+N-\/\+B\+O\+H\+M Daniel (\char`\"{}daniel.\+bensoussan-\/-\/bohm@etu.\+univ-\/lyon1.\+fr\char`\"{}) ~\newline

\item D\+O\+N\+N\+A\+Y Robin (\char`\"{}robin.\+donnay@etu.\+univ-\/lyon1.\+fr\char`\"{}) ~\newline

\item N'G\+U\+Y\+E\+N Emilien (\char`\"{}emilien.\+n'guyen@etu.\+univ-\/lyon1.\+fr\char`\"{})~\newline

\end{DoxyItemize}

Cette librairie permet de jouer au jeu A Maze Inc un plateforme/aventure/rogue-\/like ~\newline


Le code a été écrit en C++, se compile avec g++. ~\newline


Le code fonctionne normalent sous Linux et Windows, et il n'est pas nécessaire de modifier quoi que ce soit entre les deux plateformes. ~\newline


Dépendance(s) \+: ~\newline

\begin{DoxyItemize}
\item S\+D\+L2 \+: \href{http://www.libsdl.org/}{\tt http\+://www.\+libsdl.\+org/} ~\newline
 ~\newline

\end{DoxyItemize}\hypertarget{index_compil}{}\section{Pour compiler}\label{index_compil}
La compilation se fait soit avec la commande \$ make ou via codeblocks en ouvrant le projet Damibin.\+cbp puis ctrl+f11 pour recompiler tout le projet à faire seulement une fois) puis ctrl+f9. ~\newline
 Si lors de la compilation sous Windows, il y a un problème de \char`\"{}dll\char`\"{}, il faut alors copier tout les dll du dossier extern/dll/ dans bin/ ~\newline
 ~\newline
\hypertarget{index_exec}{}\section{Pour executer}\label{index_exec}
L'éxecution du jeu doit se faire dans la racine du projet puis \+: \$ ./bin/\+Jeu\+S\+D\+L, ou via codeblocks via la buildtarget Jeu\+S\+D\+L. ~\newline
 ~\newline
\hypertarget{index_doc}{}\section{Pour regénérer la documentation de code}\label{index_doc}
Dépendance \+: Doxygen \+: \href{http://www.stack.nl/~dimitri/doxygen/}{\tt http\+://www.\+stack.\+nl/$\sim$dimitri/doxygen/} ~\newline
 \$ doxygen doc/images.\+doxy ~\newline
 Puis ouvrir doc/html/index.\+html avec firefox ~\newline
 